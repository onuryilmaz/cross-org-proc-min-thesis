\chapter{INTRODUCTION}
\label{chp:introduction}

\todo{WRITE}

% process mining nedir?
process mining  
% cross-organizational process mining nedir?
cross org mining 


Process mining spectrum is extensive and highly inter-related with a different sets of activities grouped under categories of \textit{Cartography}, \textit{Auditing} and \textit{Navigation} \cite{van2011process}. In this spectrum, \textit{Cartography} activities aim to create maps of the real world activities by using process models whereas \textit{Auditing} activities confront the models and reality. Last activity group of \textit{Navigation} activities use process mining methods as navigation application. This thesis study proposes a hybrid approach and exploits approaches from different categories to create a new point of view. In other words, this study aims to discover maps using the discovery techniques from \textit{Cartography} and promotes the partially best locations in the process models from \textit{Auditing} to create recommendations for users like a \textit{Navigation} application. 

In cross-organizational process mining area, recent studies focus on commonality and collaboration between organizations; however, they only present results based on how similar the process models and behaviors of organizations under cross comparison \cite{buijs2012towards}. In addition, challenges based on partitioning of tasks and process models between organizations are presented in the literature \cite{van2011intra}. This thesis study is based on the environment where processes are executed on several organizations and cross-organizational process mining is applied with the idea of unsupervised learning where predictor variables related to performances of organizations are used.

Similarity in process mining area recent studies include various different approaches to define relationships between process models. In \cite{dijkman2011similarity}, similarity metrics are defined for node matching, structural and behavioral similarities and in \cite{buijs2014comparing} alignment matrices between models are presented. In addition, mismatch patterns are defined with the help of case studies \cite{dijkman2007mismatch}. In this study, mismatch patterns are mathematically defined and implemented; as it is known to be the first implementation of mismatch patterns. Applicability and performance of using mismatch patterns is also analyzed by comparing to the similarity metrics that are defined in the literature. 

Approach proposed in this thesis study is a four-stage solution and it starts by mining the process models of organizations with a user defined noise threshold. With the mined models and event logs, second stage calculates the performance indicators for each organization and then clusters organizations based on how well they are operating. Third stage aims to find differences between process models of each organization. Final stage combines the information from all stages and provides a set of recommendations. With this approach it is aimed to help business process management users to focus on the potentially important parts of their business maps. In addition, this approach includes implementation of mismatch patterns and performance indicator based clustering of organizations. As an extensible framework, approach stages designed with minimum dependency and they are open to include new  process mining approaches, performance indicators, clustering approaches and mismatch patterns. In addition, every stage of the methodology is intended to be configurable for user needs and business environment requirements.

Within this thesis study, approach proposed is implemented in ProM framework \cite{verbeek2010prom} as a set of plugins corresponding for each stage and packaged under the name of \textit{CrossOrgProcMin}. Since ProM is the most popular open-source environment for academia and industry at the time of this thesis published, developed set of plugins are built over this framework. With the help of this implementation, approach proposed in this thesis is tested on a synthetic and real-life event logs. Performance of methodology is tested with a defined evaluation metrics for each stage and resulting recommendations are presented to show how this approach helps users to focus on learning opportunities between organizations with a performance improvement potential. 

% rest of the thesis
Rest of the thesis is constructed as following:
...

% abstract
Cross org 
performance improvement istediklerini bunu recommendation ile yapılabileceğini yap
Process mining ile yapılanlardan (bottleneck analysis) ama buraya bakılmadığını yaz

Approach proposed in this study mines process models of organizations and calculates performance indicators. Organizations are clustered based on performance indicators and mismatches between their process models are used to generate recommendations. This approach is implemented as plugin set in ProM framework and it is designed to be extensible to include new process mining approaches, performance indicators and mismatch patterns; and configurable for user needs and business environment requirements. Proposed methodology is tested for both synthetic and real life logs for efficiency of each stage with the defined evaluation metrics and these results indicate that stages of methodology is successful and suitable within evaluation metrics. Generated recommendation results indicate that using this approach to generate recommendations significantly helps users to focus on potentially important areas of organizations process models. Recommendations not only helps user to draw attention to the differences that are difficult to spot manually and visually but also for the ones with potential for performance improvement.  
