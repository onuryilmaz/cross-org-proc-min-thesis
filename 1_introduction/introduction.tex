\chapter{INTRODUCTION}
\label{chp:introduction}

Process mining is a relatively young and developing research area that is between computational intelligence and data mining; and process modeling and analysis [X]. Main idea in this research area is to discover, monitor and improve processes by extracting information from the event logs. With this idea, process mining creates a bridge between data mining and business process modeling and analysis. Interest in this research area has two origins; firstly, events are recorded and easily available in the modern information systems. Secondly, highly competitive and rapidly changing business life requires the necessity of improving and supporting business processes [X]. Traditional process mining approaches work on single organization; however, with the increase of cloud computing and shared infrastructures, event logs of multiple organizations are currently available for analysis. In principle, there are two main environments where cross-organizational process mining stands out. Firstly, when organizations work together to handle the same processes, it is insufficient to analyze only logs of one organizations and gathered information from all stakeholders should be merged prior to analysis. Secondly, organizations essentially undertake the same processes with different needs on a common infrastructure. In such an environment, cross-organizational process mining can help the organizations to learn from each other's experience, knowledge and process execution.

Process mining spectrum is extensive and highly inter-related with a different sets of activities grouped under categories of \textit{Cartography}, \textit{Auditing} and \textit{Navigation} \cite{van2011process}. In this spectrum, \textit{Cartography} activities aim to create maps of the real world activities by using process models whereas \textit{Auditing} activities confront the models and reality. Last activity group of \textit{Navigation} activities use process mining methods as navigation application. This thesis study proposes a hybrid approach and exploits approaches from different categories to create a new point of view. In other words, this study aims to create maps using the discovery techniques from \textit{Cartography} and promotes the partially better operating locations in the process models from \textit{Auditing} to create recommendations for users like a \textit{Navigation} application. 

In cross-organizational process mining area, recent studies focus on commonality and collaboration between organizations; however, they only present results based on how similar the process models and behaviors of organizations under cross comparison \cite{buijs2012towards}. In addition, challenges based on partitioning of tasks and process models between organizations are presented in the literature \cite{van2011intra}. This thesis study is based on the environment where processes are executed on several organizations and cross-organizational process mining is applied with the idea of unsupervised learning where predictor variables related to performances of organizations are used.

Recent studies in process mining and similarity area  include various different approaches to define relationships between process models. These studies include creating similarity metrics for node matching, structural and behavioral similarities \cite{dijkman2011similarity} and alignment matrices between models \cite{buijs2014comparing}. In addition, mismatch patterns are defined with the comprehensive case studies \cite{dijkman2007mismatch}. In this study, mismatch patterns are mathematically defined and implemented; as it is known to be the first implementation of mismatch patterns. Applicability and performance of using mismatch patterns is also analyzed by comparing to the similarity metrics that are defined in the literature. 

In the light of these motivations, approach proposed in this thesis study is a four-stage solution and it starts by mining the process models of organizations with a user defined noise threshold. With the mined models and event logs, second stage calculates the performance indicators for each organization and then clusters organizations based on how well they are operating. Third stage aims to find differences between process models of each organization. Final stage combines the information from all stages and provides a set of recommendations. With this approach it is aimed to help business process management users to focus on the potentially important parts of their business maps. In addition, this approach includes implementation of mismatch patterns and performance indicator based clustering of organizations. As an extensible framework, approach stages are designed with minimum inter-dependency and they are open to include new process mining approaches, performance indicators, clustering approaches and mismatch patterns. Moreover, every stage of the methodology is intended to be configurable for user needs and business environment requirements.

Within this thesis study, proposed methodology is implemented in ProM framework \cite{verbeek2010prom} as a set of plugins corresponding for each stage and packaged under the name of \textit{CrossOrgProcMin}. Since ProM is the most popular open-source environment for academia and industry at the time of this thesis published, developed set of plugins are built over this framework. With the help of this implementation, approach proposed in this thesis is tested on a synthetic and real-life event logs. Performance of methodology is tested with a defined evaluation metrics for each stage and resulting recommendations are presented to show how this approach helps users to focus on learning opportunities between organizations with a performance improvement potential. 

Rest of the thesis is constructed as following:
\begin{itemize}
	\item In Chapter~\ref{chp:relatedwork}, related studies in process mining area are presented. 
	\item In Chapter~\ref{chp:background}, background information for the relevant topics to this thesis study is explained.
	\item In Chapter~\ref{chp:methodology}, methodology proposed in this thesis is presented with detail.
	\item In Chapter~\ref{chp:results-and-discussions}, methodology of this thesis is applied on datasets and results are mentioned.
	\item In Chapter~\ref{chp:conclusion-and-future-work}, summary of this study is presented with the final remarks and pointers for future work. 
\end{itemize}

% abstract

Process mining is a relatively young and developing research area with the main idea of discovering, monitoring and improving processes by extracting information from the event logs. With the increase of cloud computing and shared infrastructures, event logs of multiple organizations are available for analysis where cross-organizational process mining stands out for creating the opportunity for organizations learning from each other. Approach proposed in this study mines process models of organizations and calculates performance indicators; followed by clustering of organizations based on performance indicators and mismatches between their process models are used to generate recommendations. This approach is implemented as plugin set in ProM framework and it is designed to be extensible to include new process mining approaches, performance indicators and mismatch patterns; and configurable for user needs and business environment requirements. Proposed methodology is tested for both synthetic and real life logs for efficiency of each stage with the defined evaluation metrics and these results indicate that stages of methodology is successful and suitable within evaluation metrics. Generated recommendation results indicate that using this approach to generate recommendations significantly helps users to focus on potentially important areas of organizations process models. Recommendations not only helps user to draw attention to the differences that are difficult to spot manually and visually but also for the ones with potential for performance improvement.  
