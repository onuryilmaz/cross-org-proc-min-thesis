\chapter{INTRODUCTION}
\label{chp:introduction}

\todo{WRITE}

% process mining nedir?

% cross-organizational process mining nedir?
Cross-organizational mining is based on cross-correlation of workflows
and the realized activities in different organizations. The main challenge of this topic
in process mining is that comparing processes and their performances of different
organizational units in an objective approach.


% diğer çalışmalar nelere odaklandı, related work'ten?

% neler eksikti ve biz nelere odaklandık?
In process mining framework, buraya diagram tanıtıldı gibi yaz ve bizim neresine oturduğumuzu yaz

Within this framework, the study presented in this thesis is a combination of discovery
from cartography, promoting from auditing and recommendation from navigation. In
other words and within the metaphor of spectrum, this study aims to discover maps
using the discovery techniques and promotes the partially best locations in the map to
create recommendations for travelers in their navigation applications.

Notion of this thesis is based on
vertical partitioning (Process instances, namely cases, distributed over several organizations
which collaborate to complete a complex activity.) of cross-organizational process mining with the idea of unsupervised
learning where predictor variables related to performances of organizations are
used.

In this thesis, combination of metric and mismatch pattern approaches are used to identify variations between process models of different organizations.

tüm bunlar kullanılarak ortaya atılan çözüm ne oldu

implement edilerek uygulanabilirliği de test edildi.
% rest of the thesis