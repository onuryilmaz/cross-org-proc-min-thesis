\chapter{CONCLUSION AND FUTURE WORK}
\label{chp:conclusion-and-future-work}

\section{Conclusion}

In this thesis study, a new approach proposed and tested for generating recommendations using cross-organizational process mining for process performance improvement. In an environment where processes are executed on several organizations and cross-organizational process mining is applied with the idea of unsupervised learning where predictor variables related to performances of organizations are used. Results show that it is possible to use cross-organizational process mining and mismatch patterns for performance improvement recommendations. Process mining is a large-spectrum field where different set of activities and approaches are gathered together to discover, monitor and improve processes. In this thesis study, a four-stage solution is presented with each stage has different mathematical background approach and their performances are explained.

Process mining stage mine the process models of different organizations and it is shown that a generic, noise-capable process mining method can create process models with high fitness and appropriateness values. This indicates that mining process models of different organizations under a generic method can yield comparable and appropriate process models. Success of this stage directly affects the quality of calculated performance indicators since they are collected through the replay of event logs over process models.

Performance indicator analysis stage in the methodology clusters the organizations based on their performance indicators and internal evaluation metrics show that it is applicable to cluster organizations based on how well they are operating. Although there are studies that clusters organizations based on their process models for structural analysis \cite{greco2005mining}, this approach showed that organizations can be clustered based on their performance indicators.

Mismatch analysis stage in this thesis has the aim of spotting differences between processes of organizations and it is known to be the first implementation of mismatch patterns \cite{dijkman2007mismatch}. When the results of this stage is checked against well-established similarity metrics in the literature, it yields that mismatch pattern finding can be used when there is a need for spotting different approaches in similar processes. 

Recommendation generation stage collects the generated and extracted information in all prior stages to list what the organizations can learn from other organizations which perform better. In order to define performing better, different thresholds are checked for each organization and resulting recommendations show that clustering organizations based on performance indicators and then checking mismatch patterns significantly helps user to focus on the differences with a potential performance improvement. In addition, quality of recommendations in business usefulness is tried to be explained with example outputs and these examples show that the approach yields recommendations difficult to spot manually and visually, which are also potential causes of performance improvement.

In addition, proposed methodology is developed as extensible and configurable set of plugins in ProM framework \cite{verbeek2010prom} and published as open-source. This makes the methodology open to include new process mining methods, mismatch patterns and clustering approaches as well as testing with different datasets.

\section{Future Work}

For the approach proposed in this thesis study, the following issues can be listed as pointers to future work:
\begin{itemize}
	\item In the process mining stage, instead of \textit{Inductive Miner}, different techniques can be tested which can mine complex process models with higher appropriateness levels while keeping the high fitness values.
	\item In the performance indicator analysis stage, new indicators can be defined based on the business environment, event log and user needs. For instance, personnel and resource allocation indicators can be included as well as cost dimension.
	\item For mismatch pattern analysis, new and business oriented mismatch patterns can be included in the analysis. In addition analyzers can fail when there are loops in the process models in current implementations, more robust implementations for loops in process models can be implemented in the future.
	\item For the generated recommendation, their quality for business environment are not assessed within the scope of this thesis. However, when any feedback from a domain expert or BPM people is provided, the learning approach can be converted to semi-supervised learning from unsupervised learning.
	\item For ProM implementation of this study, currently user selects an organization to list recommendation; in the future user might be able to select area of interest as well as starting and ending points visually on a process model. 
\end{itemize}