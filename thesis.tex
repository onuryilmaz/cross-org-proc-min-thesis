%
% METU Institute of Natural and Applied Sciences Thesis example 
%
% Edited and Commented by Utku Erdoğdu 2013
%
% Please read the explanations so that you can customize the document		
%
% Files needed by this document:
% metu.cls 
% metu11.def (if you will use 11pt fonts) 
% metu12.def (if you will use 12pt fonts)
% metu10.def (if you will use 10pt fonts)
%
% Possible Options Here:
%
% oneandhalf, double, single : Line spacing used in the thesis. Default and institute preference is
% single.
%
% 10pt, 11pt, 12pt : Font size Default is 10pt, which is institue choice. 
% 
% pntr, pntc, pnbt : Page number position. Options are top center, top right or bottom. Default and
% institute preference is page numbers at bottom. When page numbers are at the top bottom margins
% are skewed.
% 
% chaproman, chaparabic: Chapter numbering format. Options are roman numbers and arabic numbers.
% Default is roman, institute prefers arabic
%
% oneside, twoside : Printing style. Default is twoside, which is institute choice. In this style
% chapters begin from odd numbered pages.
%
% tr, eng : Document language. This is useful if you want to translate your thesis into
% Turkish. Then you give the option tr and use \ifturkish. . .\else. . .\fi  whenever you want 
% to do something only for Turkish or only for English. Default is eng. 
% IMPORTANT!! : For official institute documents you should not use this option. 
% The Turkish format is only supplied for custom translations.
%
% ceng,aee,arme.. : You can use the abbreviated form of your department here and there is no further need to
% define the department name below. If your department name is not among the below list of defined
% departments, you should use \department and \turkishdepartment macros to define the name of your
% department.
%
% Defined Departments and Abbreviations:
% --------------------------------------
% Computer Engineering : ceng
% Aerospace Engineering : aee
% Archaeometry : arme
% Architecture : arch
% Biochemistry : bch
% Biology : biol
% Biomedical Engineering : bme
% Biotechnology : btec
% Building Science : bs
% Cement Engineering : ceme
% Chemical Engineering : che
% Chemistry : chem
% City and Regional Planning : crp
% City Planning : cp
% Civil Engineering : ce
% Computational Design and Fabrication Technologies in Architecture : arcd
% Computer Education and Instructional Technology : cte
% Design Research for Interaction : iddi
% Earthquake Studies : eqs
% Earth System Science : ess
% Electrical and Electronics Engineering : ee
% Engineering Management : em
% Engineering Sciences : es
% Environmental Engineering : enve
% Food Engineering : fde
% Geodetics - Geographical Information Technologies : ggit
% Geological Engineering : geoe
% Hydrosystems Engineering : he
% Industrial Design : id
% Industrial Engineering : ie
% Mathematics : math
% Mechanical Engineering : mech
% Metallurgical and Materials Engineering : mete
% Micro and Nanotechnology : mnt
% Mining Engineering : mine
% Operational Research : or
% Petroleum and Natural Gas Engineering : pete
% Physics : phys
% Polymer Science and Technology : pst
% Regional Planning : rp
% Restoration : rest
% Secondary Science and Mathematics Education : ssme
% Software Engineering : se
% Statistics : stat
% Structural Mechanics : st
% 
% phd, ms : Degree Received. Ph.D. or M.S. Default is M.S.
%
% End of Options
\documentclass[chaparabic,ceng,ms,12pt,oneandhalf]{metu}
% You can delete next line If your thesis does not have an appendix
\usepackage{appendix}

%
% Use your latex packages here
\usepackage{paralist}
\usepackage{amsthm}
\theoremstyle{definition}
\newtheorem{definition}{Definition}[section]
\usepackage{booktabs}
\usepackage{amssymb}

% End of Latex Packages
%
% Any personal Latex definition, decleration, etc.


% End of personal stuff
%
% Personal Information 
% ----------------------------
%
% Please check this part and fill in information about your thesis
%
% Name and Surname
\author{Onur Yılmaz}
% Thesis Title English and Turkish
\title{Recommendation Framework by Performance Clustering in Cross-Organizational Process Mining}
\turkishtitle{Çapraz Organizasyon Süreç Madenciliğinde Performansa Dayalı Kümeleme ile Öneri Sistemi}
% Department : English and Turkish
%
% Some of the departments are pre-defined, you need not redeclare them. You can use them by just
% giving an option to \documentclass. See documentation for options above. If you will define your 
% department here do not use ``Department'' or ``Bölümü'' words.
%\department{Computer Engineering}
%\turkishdepartment{Bilgisayar Mühendisliği}
%
%
% Date : You should indicate the month of your thesis defence in English.
% Default is this month
%
\date{June 2016}
%
% Approval Page Details
% --------------------------
% For each command you can give the title as optional parameter enclosed in [ ]
% This also handles the Turkish titles if you're planning to produce Turkish version of the
% document. If you'll hard code the title, you need to use turkish version of each command after
% the command itself
% 
% prof : Prof. Dr.
% assocprof : Assoc. Prof. Dr.
% assistprof : Assist. Prof. Dr.
% dr : Dr.
%
% Director of Institute
\director[prof]{Canan Özgen}
% Head of Department
\headofdept[prof]{Adnan Yazıcı}
%
% Supervisor : English and Turkish
\supervisor[assocprof]{Pınar Karagöz}
% \turkishsupervisor{  } %if you will hard-code the academic title
%
% Affiliation of Supervisor in English and possibly in Turkish
\departmentofsupervisor{Computer Engineering Department, METU}
% Co Supervisor if Any : English and Turkish
% You can just delete the next line if you don't have a co-supervisor
%\cosupervisor[assocprof]{Bedir Tekinerdoğan}
% \turkishcosupervisor{Prof. Dr. Reda Alhajj} %if you will hard-code the academic title
% Affiliation of Co-Supervisor
% You can just delete the next line if you don't have a co-supervisor
%
% Committee Members
% In general members are sorted according to their academic titles
%
% Proffesors (1)
% Associate Professors (2)
% Assistant Professors (3)
% Other (4)
% 
% IMPORTANT:  All affiliatons should fit in a single line
% If affiliation line is broken into two lines you should shorten the affiliation by using 
% abbrevations or any other means
%
% First committee member should be the chair of examining committee
% Typically the chair is one of the highest ranked committee members
% Ask your supervisor if you are not sure
\committeememberi[assocprof]{Pınar Karagöz}
\affiliationi{Computer Engineering Department, METU}
% Second committee member is always your supervisor
\committeememberii[assocprof]{Pınar Karagöz}
\affiliationii{Computer Engineering Department, METU}
% If you are an M.Sc. student and your Co-Supervisor is in your 
% examination committee, then third committee member is always your co-supervisor
%
% IMPORTANT: If you are Ph.D. student your co-supervisor can not be in your 
% examination committee.
\committeememberiii[assocprof]{Pınar Karagöz}
\affiliationiii{Computer Engineering Department, METU}
% Fourth committee member
\committeememberiv[assocprof]{Pınar Karagöz}
\affiliationiv{Computer Engineering Department, METU}
% Fifth committee member
\committeememberv[assocprof]{Pınar Karagöz}
\affiliationv{Computer Engineering Department, METU}
%
% Keywords : English & Turkish, Comma seperated
\keywords{Keyword1, Keyword2, Keyword3}
\anahtarklm{AnahtarKelime1, AnahtarKelime2, AnahtarKelime3}
%
% Abstract in English
%
\abstract{In this thesis, ..}
%
% Turkish Abstract
%
\oz{Öz...  } 
%
% Dedication 
\dedication{\textit{To my family and people who are reading this page}}
%
%
% Acknowledgements   
\acknowledgments{
I would like to thank my supervisor ...
}
%
% End of Personal and Introductory Information
%%%%%%%%%%%%%%%%%%%%%%%%%%%%%%%%%5

%%% !!! This two should be last lines before \begin{document}, do no move them !!!
\usepackage[pdftex]{hyperref}
\usepackage[all]{hypcap}
\usepackage{todonotes}
\usepackage{graphicx}
\graphicspath{ {./images/} }
\usepackage{rotating}
\usepackage{xy} 
\usepackage{booktabs}
\usepackage{pifont}
\usepackage{color}
\usepackage{listings}
\usepackage{pdfpages}
\usepackage{array}
\newcolumntype{M}{>{\centering\arraybackslash}m{\dimexpr.25\linewidth-2\tabcolsep}}
\renewcommand\lstlistingname{XChor Language - }
\def\lstlistingautorefname{XChorCode.}
\lstset{
language = java,
basicstyle=\small,
numbers=left,
numbersep=10pt,
numberstyle=\tiny\color{black},
stepnumber=1,
tabsize=2,
showspaces=false, 
frame=single,
breaklines=true,
escapeinside=~~
}
\usepackage{float}
\restylefloat{figure}
\newcommand{\tab}{\hspace*{2em}}
\DeclareGraphicsExtensions{.pdf,.png,.jpg}

\begin{document}
% Preliminaries
\begin{preliminaries}
% If you are willing to use any custom stuff before Chapters, put it here
% Such as List of Abbreviations
% Check the abbreviations.tex for a template list of abbreviations

\begin{theglossary}{LONGESTABBRV}
\item[BPM] Business Process Management
\item[BPMN] Business Process Modeling Notation
\item[ERP] Enterprise Resource Planning
\item[OMG] Object Management Group
\item[SBPMI] Shared Business Process Management Infrastructure
\end{theglossary}

% End of Preliminaries
\end{preliminaries}
%   
% Latex content Goes Here 
% 
%
\chapter{INTRODUCTION}
\label{chp:introduction}

Process mining is a relatively young and developing research area with the roots in computational intelligence, data mining; and process modeling and analysis \cite{van2012process}. Main idea in this research area is to discover, monitor and improve processes by extracting information from event logs. With this idea, process mining creates a bridge between data mining and business process modeling and analysis. Interest in this research area has two origins; firstly, events are recorded and easily available in the modern information systems. Secondly, highly competitive and rapidly changing business life requires improvement and support for business processes \cite{van2012process}. Traditional process mining approaches work on a single organization; however, with the increase of cloud computing and shared infrastructures, event logs of multiple organizations are currently available for analysis. In principle, there are two main environments where cross-organizational process mining stands out. Firstly, when organizations work together to execute the same process, it is insufficient to analyze only logs of one organization and gathered information from all stakeholders should be merged prior to analysis. Secondly, organizations essentially execute the same processes with different needs and configurations on a common infrastructure, where cross-organizational process mining can help the organizations to learn from each other's experience, knowledge and processes.

Process mining spectrum is extensive and highly inter-related with a different sets of activities grouped under categories of \textit{Cartography}, \textit{Auditing} and \textit{Navigation} \cite{van2011process}. In this spectrum, \textit{Cartography} activities aim to create maps of the real world tasks by using process models whereas \textit{Auditing} activities confront the models and reality; and the last activity group of \textit{Navigation} activities use process mining methods like a navigation application. This thesis study proposes a hybrid approach and exploits approaches from different categories to create a new point of view. In other words, this study aims to create maps using the discovery techniques from \textit{Cartography} and promotes the better operating locations in the process models from \textit{Auditing} to create recommendations for users like a \textit{Navigation} application. 

In cross-organizational process mining area, recent studies focus on commonality and collaboration between organizations; however, they only present results based on how similar the process models and behaviors of organizations under cross comparison \cite{buijs2012towards}. In addition, challenges based on partitioning of tasks and process models between organizations are presented in the literature \cite{van2011intra}. This thesis study is based on the environment where processes are executed on several organizations and cross-organizational process mining is applied with the idea of unsupervised learning where predictor variables related to performances of organizations are used.

Recent studies in process mining and similarity area  include various different approaches to define relationships between process models. These studies include creating similarity metrics for node matching, structural and behavioral similarities \cite{dijkman2011similarity} and alignment matrices between models \cite{buijs2014comparing}. In addition, differences between process models are tried to be explained by mismatch patterns with the help of comprehensive case studies \cite{dijkman2007mismatch}. In this study, mismatch patterns are mathematically defined and implemented; as it is known to be the first implementation of mismatch patterns. Applicability and performance of using mismatch patterns is also analyzed by comparing to the similarity metrics that are defined in the literature. 

In the light of these motivations, approach proposed in this thesis study is a four-stage solution and it starts by mining the process models of organizations with a user defined noise threshold. With the mined models and event logs, second stage calculates the performance indicators for each organization and then clusters organizations based on how well they are operating. Third stage aims to find differences between process models of each organization. Final stage combines the information from all stages and provides a set of recommendations. With this approach it is aimed to help business process management users to focus on the potentially important parts of their business maps. In addition, this approach includes implementation of mismatch patterns and performance indicator based clustering of organizations. As an extensible framework, approach stages are designed with minimum inter-dependency and they are open to include new process mining approaches, performance indicators, clustering approaches and mismatch patterns. Moreover, every stage of the methodology is intended to be configurable for user needs and business environment requirements.

Within this thesis study, proposed methodology is implemented in ProM framework \cite{verbeek2010prom} as a set of plugins corresponding for each stage and packaged under the name of \textit{CrossOrgProcMin}. Since ProM is the most popular open-source environment for academia and industry at the time of this thesis published, developed set of plugins are built over this framework. With the help of this implementation, approach proposed in this thesis is tested on a synthetic and real-life event logs. Performance of methodology is assessed with a defined evaluation metrics for each stage and resulting recommendations are presented to show how this approach helps users to focus on learning opportunities between organizations with a performance improvement potential.

Rest of the thesis is constructed as following:
\begin{itemize}
	\item In Chapter~\ref{chp:relatedwork}, related studies in process mining area are presented. 
	\item In Chapter~\ref{chp:background}, background information for the relevant topics to this thesis study is explained.
	\item In Chapter~\ref{chp:methodology}, methodology proposed in this thesis is presented with detail.
	\item In Chapter~\ref{chp:results-and-discussions}, methodology of this thesis is applied on datasets and results are discussed.
	\item In Chapter~\ref{chp:conclusion-and-future-work}, summary of this study is presented with the final remarks and pointers for future work. 
\end{itemize}


\chapter{RELATED WORK}
\label{chp:relatedwork}

In this chapter, studies related to the work presented in this thesis are presented. Firstly, latest trends and studies in the process mining area are explained. Then studies from cross-organizational process mining, which is the main topic of this research, is introduced. Following these, studies related to similarity in process mining is mentioned. Finally, performance and conformance analysis approaches in process mining area are presented. In each section, survey of related studies are aimed to be limited with the subject of this thesis.

\section{State of the Art in Process Mining}
\label{sec:state-of-the-art-in-process-mining}

In this section, important milestones in process mining and current research trends will be presented. Studies in process mining area have roots based on process discovery techniques for software engineering. These original techniques are studied to discover workflows using neural networks, purely algorithms and Markovian approaches \cite{van2004process}. However, using process mining in the workflow management area is introduced by Agrawal et al. and this study aims to find workflow graphs given event logs and identify edge conditions between nodes \cite{agrawal1998mining}. In the following efforts, various different approaches based on hierarchical structuring, dependency and frequency graphs; and heuristic methods are suggested to address the same problem. While some of these algorithms are focused on creating partial solutions, some of the algorithms are used as a foundation for further expansions such as "Alpha Algorithm" of van der Aalst et al.\cite{van2004workflow}. Spread of process mining is not only in kept in academia, but also many vendors presented tools and software to discover processes and help organizations manage their workflows \cite{accorsi2014unleashing}.

\begin{figure}
  \centering
  \includegraphics[width=\textwidth]{2_relatedwork/process-mining-spectrum-v2}
  \caption{Process Mining Framework}
  \label{fig:process-mining-spectrum-v2}
\end{figure}

Current challenges in the process mining area can be presented best through mentioning spectrum and its relations. As diagrammed in Figure~\ref{fig:process-mining-spectrum-v2} and presented in the research of van der Aalst \cite{van2011process}, process mining spectrum is extensive and highly inter-related. Starting with the provenance, process mining deals with gathering and keeping event logs as \textit{current data} and \textit{historic data}. In this level, \textit{post mortem data} means the logs and information gathered from completed events whereas \textit{pre mortem data} means information related to ongoing cases. This separation helps us to use historical knowledge to exploit and gain advantage over current business operations. At the very bottom of the diagram, models are presented as \textit{de facto models} and \textit{de jure models}. \textit{De jure models} are created with the aim of describing reality as it should be whereas \textit{de facto models} aims to describe reality as it is. Within each model category a mixture of perspectives, in other words point of views from different information levels, can be mixed. In order to fill the gap between event logs and models, ten different process mining activities are listed under three categories \cite{van2011process} as follows:

\begin{description}
\item[Cartography:] In this category the main aim is to create maps of the real world activities by using process models. 
	\begin{description}
	\item[Discovery:] Discovery is focused on extracting process models from event logs. 
	\item[Enhancing:] Enhancing activities are based on improving \textit{de facto models} with the information hidden in the event logs. 
	\item[Diagnose:] Diagnose activities stand for identifying the problems caused by directly process models.
	\end{description}

\item[Auditing:] In this category, activities related to confronting the model and the reality are collected.
	\begin{description}
	\item[Detect:] Detection is based on comparing the \textit{de jure models} with the ongoing processes to generate alerts if shifts occur in the reality.
	\item[Check:] Checking is defined to identify deviations occurred in the past.
	\item[Compare:] Comparison of \textit{de facto models} and \textit{de jure models} shows the difference between planned and the reality.
	\item[Promote:] Promotion includes  partially updating \textit{de jure models} from the information gathered from \textit{de facto models}.
	\end{description}

\item[Navigation:] In this category, process mining methods are used like a navigation application.
	\begin{description}
	\item[Explore:] Event data and models are used in combination to explore business processes.
	\item[Predict:] Prediction activities combine the information from past events and make predictions about ongoing processes.
	\item[Recommend:] Suitable actions can be suggested by using the prediction and contextual  information.
	\end{description}
\end{description}

Within this framework, the study presented in this thesis is a combination of discovery from cartography, promoting from auditing and recommendation from navigation. In other words and within the metaphor of spectrum, this study aims to discover maps using the discovery techniques and promotes the partially best locations in the map to create recommendations for travelers in their navigation applications. 

\section{Process Discovery in Process Mining}
\label{sec:process-discovery-in-process-mining} 
\todo{https://hal.archives-ouvertes.fr/hal-00803968/document 2.3 ile burayı genişlet}   

\section{Cross-organizational Process Mining}
\label{sec:cross-organizational-process-mining}
In this section, related studies and trends in cross-organizational process mining will be presented. Cross-organizational mining is based on cross-correlation of workflows and the realized activities in different organizations. The main challenge of this topic in process mining is that comparing processes and their performances of different organizational units in an objective approach. This objective approach is open to be enhanced with the process context, namely the environment of the process that is executed. Most of the studies in this area are currently studying to reveal the possible opportunities and some initial approaches to address main challenges.

In the study of Bujis et al. \cite{buijs2012towards}, the authors indicated the importance of the increase of Software-as-a-Service (SaaS) and cloud computing infrastructure usages. As a result of this increase, more and more organizations will use a Shared Business Process Management Infrastructure (SBPMI) and it is an opportunity for different organizational units to learn from each other in such infrastructure. The approach presented in this study is based on three questions and three metric groups to answer these questions:
\begin{enumerate}
\item Which organizations support my behavior with better process models?
\item Which organizations have better behavior which my process model supports? and
\item Which set of organizations can I support with my process model?
\end{enumerate} While answering these questions, they used \textit{simplification based metrics} to indicate better process models; and \textit{throughput time metrics} to indicate better behaviors. In this study, it is shown how a generic framework can be used to highlight the main idea behind cross-organizational process mining. 


In the study of van der Aalst \cite{van2011business}, using configurable process models is proposed for the organizations sharing the same infrastructure and doing the similar work. Configurable process models are defined as a family of process models where each organization can use this family with their configurations according to their business needs. This approach not only creates behaviors needed by each organization but also creates a basis to compare and learn within process mining framework. The configurable process models are formalized by "Causal Nets", which is a notational language based on input and output bindings of each node. Although this study does not answer to all challenges related to cross-organizational-process-mining, a formalism of configurable processes models is presented to address learning and conformance checking.

Cross-organizational process mining is divided into intra-organizational and inter-organizational with two basic ideas: \textit{collaboration} and \textit{exploiting} commonality \cite{van2011intra}. In his study, van der Aalst defined collaboration for distributed work between multiple organizations and commonality as ability of using same process models and infrastructure between the organizations. In order to exploit these two ideas, two partitioning dimensions are suggested in \cite{van2011intra}:

\begin{description}
	\item[Vertical Partitioning:] Process instances, namely cases, distributed over several organizations which collaborate to complete a complex activity. 
	\item[Horizontal Partitioning:] Process parts, namely tasks or activities, are shared within organizations like jigsaw puzzle parts.
\end{description}

For these orthogonal dimensions, a number of questions and challenges are presented in the study  \cite{van2011intra}. For the vertical partitioning, where organizations are aiming to share infrastructure and knowledge to learn from each other, it is mentioned that supervised learning methods like classification can be used. Notion of this thesis is based on vertical partitioning of cross-organizational process mining with the idea of unsupervised learning where predictor variables related to performances of organizations are used.
 
\section{Process Similarity in Process Mining}
\label{sec:process-similarity-in-process-mining}

In this section, prominent studies related to similarity in process mining will be presented with their results. With the emerging attention in business processes, organizations become aware of the fact that they can exploit the business processes and their similarities \cite{buijs2014comparing}. In addition, most of the large organizations have repository of process models of similar business operations \cite{dijkman2011similarity}. There are three main different approaches proposed to this solution in the literature currently. 

The first approach, proposed by Dijkman et al. \cite{dijkman2011similarity}, is based on similarity metrics as 
\begin{inparaenum}[\itshape a\upshape)]
\item node matching similarity;
\item structural similarity; and
\item behavioural similarity.
\end{inparaenum}
Result of the study \cite{dijkman2011similarity} indicates that using these three similarities can differentiate comparable process models and within these metrics structural similarity is the most prominent one. 

The second approach by Bujis and Reijis is an analytical approach \cite{buijs2014comparing} to compare the process models of different organizations that does similar works. Proposed algorithm is based on creating an alignment matrix between observed and realized models. This inter-relation is also used to compare different variants of the same process by different organizations. In addition, they suggested their method as a framework to further standardize a process of common interest. 

The third and final approach is based on frequently occurring mismatches between similar business processes \cite{dijkman2007mismatch}. In this study \cite{dijkman2007mismatch}, a set of mismatch patterns are derived from the different department of the same organization. Although this research does not present a complete set of possible mismatch patterns, the provided set is comprehensive to identify similarities and differences of the processes of same operations. In this thesis, combination of metric and mismatch pattern approaches are used to identify variations between process models of different organizations.

\section{Performance and Conformance Analysis in Process Mining}
\label{sec:performance-and-conformance-analysis-in-process-mining}
In this section summary of studies related to performance and conformance analysis will be presented. Event logs do not only contain task sequences but also contain time, resource and contextual information. Analysis of process models with these additional information is used within performance analysis framework to highlight bottlenecks or make predictions. However, in order to undertake a performance analysis there is a need of replaying reality (event logs) on the expectation (process models) and check the conformance of reality to plan \cite{van2012replaying}. 

In the study conducted by Rozinat and van der Aalst \cite{rozinat2008conformance}, a conformance framework is proposed by two metrics \textit{fitness metrics} and \textit{appropriateness metrics}. In this study \cite{rozinat2008conformance}, fitness metric is based on replaying the event log on the process model and counting the number of missing or remaining tokens. In other words, replaying and conformance of event logs over process models is modelled as a token passing formalism. On the other hand, appropriateness metrics are based on how accurate the process model in describing reality within a degree of clarity. Simplicity, precision and generalization attributes of the process models are taken into account while calculating appropriateness metrics.

Token passing approach to conformance has a major drawback when the process model and reality of event logs do not fit completely. In this case, there are overestimated process models that are too general, in other words with too much behavior other than the reality. In order to overcome this drawback, heuristic and optimization based methods are proposed by other researchers. In this thesis study, an extended version of optimization based approach presented in \cite{adriansyah2011conformance} is used to replay logs on the process models. Since the main goal of this study is not to evaluate conformance of logs and process models, the method presented by Adriansyah et al. is extended to calculate performance indicators while replaying the logs.
% CHAPTER 3

\chapter{BACKGROUND}
\label{chp:background}
In this chapter background information is presented for the relevant topics to this thesis study. Topics are mentioned starting with the building blocks and then methodologies with the order of their usage in the methodology presented in this thesis study. Firstly, event logs and process models are mentioned with their mathematical formalization and background. Following these, process discovery approaches are explained and then performance analysis methodologies are presented. Then, machine learning and clustering topics are mentioned. Finally, mismatch patterns in process models are presented. All topics in this chapter are limited to the scope of this thesis study with the aim of constructing a necessary background.

\section{Event Log}
\label{sec:event-log}
Event logs are the main inputs to any process mining methodology including this thesis study and they include information related to real life activities. Event logs which are the outputs of the software systems like Enterprise Resource Planning (ERP) or Business Process Management (BPM) have common properties that are also assumed in the literature as the properties of event logs. General structure of event logs includes multiple layers as diagrammed in Figure~\ref{fig:event-log-structure}. Processes have cases which are simply single process instances. Within each case, there are events that are generally represents a sequence of activities performed. Each event is enhanced with the attributes such as timestamps, resource assignments and other contextual data. A fragment of this structure is presented in Table~\ref{table:event-log-loan} for the Loan Application Process\cite{loan-app-data}, which shows the footprints of a financial organization that provides consumer credits \cite{buijs2013improving}. In the table, each line represents an event with its attributes which are collected under cases to form a complete event log. 
\begin{figure}
  \centering
  \includegraphics[width=\textwidth]{3_background/event_log_structure}
  \caption{Structure of Event Logs}
  \label{fig:event-log-structure}
\end{figure}

\begin{table}[h]
\centering
\caption{Fragment of event log from Loan Application Process (Variation #1)\cite{loan-app-data}}
\label{table:event-log-loan}
\begin{tabular}{@{}llccc@{}}
\toprule
\multicolumn{5}{c}{\textbf{Event Log}}                                                                  \\ \midrule
                        &                         & \multicolumn{3}{c}{\textbf{Attributes}}             \\ \midrule
                        & \textbf{Event}          & \textbf{Date} & \textbf{Time} & \textbf{Transition} \\ \midrule
\textbf{Case \#1}       & Register Application    & 16.04.2013    & 14:37:27      & Complete            \\ \midrule
                        & Check Credit            & 16.04.2013    & 14:41:19      & Complete            \\ \midrule
                        & Check System            & 16.04.2013    & 14:47:35      & Complete            \\ \midrule
                        & Calculate Capacity      & 16.04.2013    & 14:50:21      & Complete            \\ \midrule
                        & Accept                  & 16.04.2013    & 14:53:22      & Complete            \\ \midrule
                        & Send decision e-mail    & 16.04.2013    & 14:55:11      & Complete            \\ \midrule
\textbf{Case \#2}       & Register Application    & 16.04.2013    & 16:28:19      & Complete            \\ \midrule
                        & Check Credit            & 16.04.2013    & 16:36:22      & Complete            \\ \midrule
                        & Check System            & 16.04.2013    & 16:43:10      & Complete            \\ \midrule
                        & Calculate Capacity      & 16.04.2013    & 16:52:40      & Complete            \\ \midrule
                        & Reject                  & 16.04.2013    & 16:53:53      & Complete            \\ \midrule
                        & Send decision e-mail    & 16.04.2013    & 17:01:32      & Complete            \\ \midrule
\multicolumn{1}{c}{...} & \multicolumn{1}{c}{...} & ...           & ...           & ...                 \\ \bottomrule
\end{tabular}
\end{table}

Event log structure and related notions are formalized in \cite{van2011process} as following:
  \theoremstyle{definition}
  \begin{definition}{}
  (Event and Event Attributes) Let \textit{E} be the universe of events which include all possible event identifiers and in this universe any event \textit{e} $\in$ \textit{E}. Events are enhanced with contextual information, namely attributes. For any event \textit{e} $\in$ \textit{E} and attribute \textit{A}, $\#_\textit{A}(\textit{e})$ is the value of the attribute \textit{A} for event \textit{e}. Possible attributes for events include timestamps, people and resource assignments, transaction types and other contextual data.
  \end{definition}
  \theoremstyle{definition}
  \begin{definition}{}
  (Case and Case Attributes) Let \textit{C} be the universe of cases which include all possible case identifiers and in this universe any case \textit{c} $\in$ \textit{C}. Like events, cases are also enhanced with contextual information, namely attributes. For any case \textit{c} $\in$ \textit{C} and attribute \textit{A}, $\#_\textit{A}(\textit{c})$ is the value of the attribute \textit{A} for case \textit{c}. Each case has at least one attribute which is trace.
  \end{definition}
  \theoremstyle{definition}
  \begin{definition}{}
  (Trace) Trace is a sequence of event \textit{t} $\in$ ${E}^{*}$ such that each event is restricted to occur only once.
  \end{definition}
  \theoremstyle{definition}
  \begin{definition}{}
  (Event Log) Event log is set of cases \textit{L} $\subseteq$ \textit{C} such that each event occurs at most once in the event log.
  \end{definition}

In this study, event logs from different organizations are exploited, thus organization related attributes are used for cases. Within these event logs, traces of cases for each organization are used to discover underlying process models. In addition, timestamps and resource related attributes of events are used to collect performance related data for further analysis.

\section{Process Modeling}
\label{sec:process-modeling}
Process modeling is the foundation of process management applications and main tools of people in this profession. Although process modeling can be defined with informal process workflows to document procedures, there are a number of formalized notations which are more suitable to cross-applicability and mathematical analysis. In the control-flow view of process modeling, a process model is aimed to give decisions on which activities to take place with their orders. In this study, control-flow of process models are used to find mismatch patterns between different organizations that execute the same activities. Considering the scope of this  thesis, only Petri nets, Workflow Nets and Business Process Modeling Notation (BPMN) will be presented in this section. 

\subsection{Petri Nets}
\label{sec:petri-nets}
Petri net is a mathematical modeling language that is aimed to describe concurrent systems. Graphical notation of Petri nets seems intuitive and simple; however, it is powerful in terms of being executable and applicability of analysis techniques \cite{vanderAalst:2011:MBP:2000715}. Petri nets are directed bipartite graphs where \textit{nodes} represent transitions and \textit{places} represent conditions. Structure represented by Petri nets is static and the state of the net is described by placing \textit{tokens}, namely the process of \textit{marking}. Formalization of Petri nets are explained in \cite{reisig1998lectures} as following:
\theoremstyle{definition}
\begin{definition}{}
(Petri Nets) A \textit{Petri net} is a triplet $N = (P, T, F)$ where $P$ is finite set of \textit{places}, $T$ is finite set of \textit{transitions} and $F$ is set of \textit{flow relations} where:
\begin{enumerate}
  \item (Separation) $P \cap T = \varnothing$
  \item (Flow relation) $F \subseteq (P \times T) \cup (T \times P)$
\end{enumerate}
\end{definition}

\subsection{Workflow Nets}
\label{sec:workflow-nets}
Process models in the real life have additional properties to be executable and they are defined in the Workflow net formalization, which is simply a subset of Petri nets. These additional properties can be formalized as following \cite{van2013discovering}:
\theoremstyle{definition}
\begin{definition}{}
(Workflow Nets) Let $N = (P, T, F)$ be a Petri net and $t$ is a new identifier not in $P \cup T$. $N$ is a workflow net (WF-net) if and only if:
\begin{enumerate}
  \item (Start Node) $P$ contains a \textit{source place i} where no token can be fired to.
  \item (End Node) $P$ contains a \textit{sink place o} where no token can be fired from.
  \item (Connectedness) $\bar{N} = (P, T \cup \{t\}, F \cup \{(o,t),(t, i)\})$ is strongly connected, in other words there is a directed path between any pair of nodes in $\bar{N}$.
\end{enumerate}
\end{definition}

In simple terms, a Workflow net is a Petri net with a source place to start the process and a sink place to end; furthermore, all nodes are on a path from source place to sink place \cite{van1998application}.  In order to illustrate this formalization, the Workflow net for the event log mentioned in Section~\ref{sec:event-log} is presented in Figure~\ref{fig:loan-petri-net}. In the figure, places are indicated by circles, transitions are indicated by rectangles, and flow relations are represented by arcs. 
\begin{figure}
  \centering
  \includegraphics[width=\textwidth]{3_background/loan-petri-net}
  \caption{Workflow net of Loan Application Process (Variation #1)}
  \label{fig:loan-petri-net}
\end{figure}

Workflow nets are representatives of real life processes; however, they can result with processes including deadlocks, live-locks and never-reached activities. In order to avoid process models from these problems, soundness definition is suggested in \cite{van1998application} and it is simplified as following with the context of this thesis:
\theoremstyle{definition}
\begin{definition}{}
(Soundness) Let $N$ be a Workflow net and it is sound if and only if:
\begin{enumerate}
  \item  (Safeness) Places cannot hold more than one tokens at the same time.
  \item  (Proper completion) Any marking of net can reach to sink place.
  \item  (Absence of dead tasks) Net does not contain any dead transitions.  
\end{enumerate}
\end{definition}

In this thesis, Workflow nets are used to discover and present underlying process models of different organizations. Considering the applicability of well-known process mining algorithms on Workflow nets and implementations in ProM Framework \cite{verbeek2010prom}, Workflow nets are used as the notation for discovery and analysis.

\subsection{Business Process Modeling Notation (BPMN)}
\label{sec:bpmn} 
Business Process Modeling Notation (BPMN) is one of the most popular and widely used modeling language implemented by many vendors. In addition to its popularity, this notation is standardized by the Object Management Group (OMG) since 2004. In this notation, atomic activities are named as \textit{tasks} and routing decision logic is implemented by \textit{gateways}. These gateways include split and join gateways with AND, OR and XOR logic operations. In addition, deferred choice pattern is implemented by \textit{event-based XOR gateway} in BPMN to handle race conditions between tasks that are running parallel \cite{van2003workflow}. Since the primary goal of BPMN is to provide a standardized notation that is easy to understand by business stakeholders, in this study resulting nets are converted to BPMN diagrams for visual analysis by the plugin implemented in ProM \cite{kalenkovaprocess}. BPMN diagram of the Workflow net from Figure~\ref{fig:loan-petri-net} is presented in Figure~\ref{fig:loan-bpmn}. As can be seen from the diagram, gateways help to understand the relations and dependencies of the tasks.

\begin{figure}
  \centering
  \includegraphics[width=\textwidth]{3_background/loan-bpmn}
  \caption{Process Model of Loan Application Process  (Variation #1) using BPMN}
  \label{fig:loan-bpmn}
\end{figure}


\section{Process Discovery}
\label{sec:process-discovery}
In process mining field, one of the most challenging task is to construct a process model based on the behavior in the event logs, namely process discovery. Many process discovery algorithms are proposed to address different challenges in process discovery and using different notations. However, in this thesis focus of the study is learning from the cross-organizational process mining rather than address all process discovery challenges. With this reasoning, Inductive Process Mining \cite{leemans2013discovering} is selected as appropriate since it is simple, highly applicable and configurable. In the literature, its derivatives which handles infrequent behaviors \cite{leemans2014discoveringinfrequent}; incomplete logs \cite{leemans2014discoveringincomplete}; and model optimization \cite{weidlich2012profiles} are also available.  

\textit{Inductive Miner (IM)}, which is proposed as an extensible framework in \cite{leemans2013discovering}, aims to discover block-structured process models that are sound and well-fitting to the behavior represented in event log. In addition, this approach focuses on creating rediscoverable models that is a key attribute in this study. Formalization of the key points in the study \cite{leemans2013discovering} are as following:

\theoremstyle{definition}
\begin{definition}{}
(Block-structured Workflow Nets) Block-structured WF-nets are subset of WF-nets where the workflow can be divided recursively into  parts with single entry and exit points.  
\end{definition}

\theoremstyle{definition}
\begin{definition}{}
(Rediscoverability) Let a process is expressible by a model $M$ which is unknown priori. Given a log $L$ of $M$, $L$ is a subset of language used to describe model $M$. $M$ is isomorphic-rediscoverable from $L$ by mining algorithm $B$ if and only if $M \in B(L)$.
\end{definition}
 
Framework developed in the study \cite{leemans2013discovering} uses a divide-and-conquer approach to discover subprocesses of sublogs obtained by splitting the event log. Main steps of the algorithm are listed as following:
\begin{enumerate}
  \item Activity Sets: Split the \textit{activities} in log to disjoint sets.
  \item Sublogs: Split the \textit{log} by using \textit{activity sets}.
  \item Recursive Mining: Mine sublogs with these steps until a sublog contains only single activity.
\end{enumerate}

In the study \cite{leemans2013discovering}, an algorithm based on this framework is presented that guarantees returning a sound and fitting process model in finite time. Therefore, this framework is selected as appropriate and its extension that can handle infrequent behavior is used within this thesis study to address challenges of different event logs. In real life, most of the cases in the events are samples of frequent behavior; however, there are also infrequent behaviors according to the nature of process execution environment. In real life, these different paths are used infrequently; however, their effect in discovery is still significant. \textit{Inductive Miner Infrequent (IMi)} is proposed in \cite{leemans2014discoveringinfrequent} as an extension to \textit{Inductive Miner} to handle noise in the event logs. By filtering the infrequent behavior, it is aimed that \textit{IMi} succeeds with improved models discovered. After each recursive step of \textit{IM}, filtering is applied if the discovered model is a flower model which represents infrequent behavior. Basic idea behind filtering is setting a user-defined threshold between 0 to 1 and with the help of this threshold, both log splittings and mining operations are done on a cleaner subset of logs. When the discovered models compared to \textit{IM}, \textit{IMi} results with lower fitness, higher precision and equal generalization.
In this thesis study, infrequent behavior capable implementation is used for experimenting the cross-organizational mining of process models with their implementation of ProM framework \cite{verbeek2010prom}.

\section{Process Performance Analysis}
\label{sec:process-performance-analysis}
In the main and traditional tasks of process mining spectrum, event logs are used to discover and enhance process models. In addition to these main tasks, process mining enables to discover relationships between event logs and process models for conformance and performance analysis. Within conformance, any deviations from modeled behavior can be discovered, moreover when the logs are replayed on the models bottleneck analysis can be undertaken with the help of timestamps on the event logs. However, in order to replay event logs on the process models there is a need for \textit{alignment} which is formalized in \cite{van2012replaying}. Formalization and notions presented in the study \cite{van2012replaying} are based on the assumption that process models and event logs use the same set of activity labels and therefore they are can be related.

The first notion in alignment of process model and event log is defining the relationship between \textit{moves in the model and log}. The necessity of this notion is based on the fact that some moves in the event log cannot be operated with the process model and vice versa. In the study \cite{van2012replaying}, move and alignment are defined as following:

\theoremstyle{definition}
\begin{definition}{}
(Move) For the event log $L$, $A_{L}^{\bot} = A_{L} \cup \{ \bot\}$ is defined where $x \in A_{L}$ refers to \textit{"move x in log"} and $\bot$ refers to \textit{"no move in log"}. One step in alignment is represented as $(x,y) \in A_{L}^{\bot} \times A_{M}^{\bot}$ such that:
\begin{enumerate}
  \item $(x,y)$ is a \textit{move in log} if $ x \in A_{L}$ and $y=\bot$,
  \item $(x,y)$ is a \textit{move in model} if $x=\bot$ and $y \in A_{M}$,
  \item $(x,y)$ is a \textit{move in both} if $x \in A_{L}$ and $y \in A_{M}$,
  \item $(x,y)$ is an \textit{illegal move} if $x=\bot$ and $y=\bot$.
\end{enumerate}
In this environment, \textit{legal moves} are defined as $A_{LM} = \{ (x,y) \in A_{L}^{\bot} \times A_{M}^{\bot} |  x \in A_{L} \vee y \in A_{M} \}$
\end{definition}

\theoremstyle{definition}
\begin{definition}{}
(Alignment) Let $\sigma_{L} \in L$ a trace in event log $L$ and let $\sigma_{M} \in \beta (M)$ a full execution sequence of process model $M$. An alignment of $\sigma_{L}$ and $\sigma_{M}$ can be defined as a sequence $\gamma \in {A_{LM}}^{*}$  where the projection of first element yields $\sigma_{L}$ and the projection of second element yields $\sigma_{M}$. 
\end{definition}

In order to qualify the alignment operations, distance function on legal moves is defined in \cite{van2012replaying} as following:
\theoremstyle{definition}
\begin{definition}{}
(Distance Function) Distance function is defined on legal moves as $\delta \in A_{LM} \rightarrow \mathbb{N}$ to associate costs to moves in alignment:
\begin{enumerate}
  \item If $x \in A_{L}$ and $y=\bot$, then  $\delta(x,y)$ is the cost of \textit{move x in log}.
  \item If $x=\bot$ and $y \in A_{M}$, then  $\delta(x,y)$ is the cost of \textit{move y in model}.
  \item If $x \in A_{L}$ and $y \in A_{M}$, then  $\delta(x,y)$ is the cost of \textit{move x in log and move y in model} (This cost is generally $\delta(x,y) = 0$ if $x = y$).
\end{enumerate}
\end{definition}

According to this distance function definition, various different functions can be defined using costs. For instance, in \cite{van2012replaying} a \textit{standard distance function} is defined as no cost when log and model agree and cost of 1 otherwise. In addition, optional alignment between process model and event logs is defined as following:
 
\begin{definition}{}
(Optimal Alignment) Let $\sigma_{L} \in L$ be a trace in event log $L$, $M$ a process model and $\Gamma_{\sigma_{L},M} = \{ \gamma \in {A_{LM}}^{*} \mid \exists_{\sigma_{M} \in \beta (M)}\ \gamma\ is\ an\ alignment\ of\ \sigma_{L}\ and\ \sigma_{M} \}$. An alignment $\gamma \in \Gamma_{\sigma_{L},M}$ is \textit{optimal} for event log trace $\sigma_{L}$ and model $M$ if for any ${\gamma}' \in \Gamma_{\sigma_{L},M} : \delta({\gamma}')\geq \delta(\gamma)$.
\end{definition}

This optimal alignment can be found with the help of different approaches and within process mining field, proposed methods \cite{adriansyah2011conformance} \cite{adriansyah2011towards} are based on $A^{*}$ algorithm which is a path-finding algorithm based on graphs. In principle, any optimization methodology can be used to find the optimal alignment and in this thesis study an $A^{*}$ based implementation in ProM Framework \cite{verbeek2010prom} is used to find optimal alignments. Although this thesis study has no direct focus on conformance of process models by event logs, these steps were necessary to replay the event logs over process models. With the help of replay, any performance indicator can be calculated to compare performances of cross-organizational processes. Using the timestamp information or resource utilization in the event logs, performance indicators can be discovered while replaying the log after alignment. These performance indicators can include lead time, service time, waiting time in \textit{time dimension}; and utilization or activity costs in \textit{cost dimension} \cite{van2011process}.

\section{Machine Learning and Clustering}
\label{sec:unsupervised-learning}
Machine learning is a study area of computer science which have roots in pattern recognition and computational learning theory of artificial intelligence. Machine learning approaches work on construction and learning from data to make further predictions. There are various approaches proposed in this area and clustering approach will be presented in this section. Cluster analysis is based on assigning the set of observations into subsets (\textit{clusters}) so that observations within the same cluster are similar whereas the observations from different clusters are dissimilar, where the similarity criteria is predefined. Clustering is a method in unsupervised learning, where the main problem is learning a hidden structure in unlabeled data. Since the provided data is unlabeled there is no error or reward assignment to the potential solutions provided these approaches; however, quality metrics on clusters are used to evaluate results. With these characteristics, clustering is a common technique in exploratory data mining and statistical data analysis and it is used in many fields like image analysis, information retrieval and bioinformatics.

In cluster analysis, various algorithms are proposed with different approaches on defining clusters and how to efficiently find them. Popular approaches are based on the idea of decreasing the distance among the members of same cluster, space density of data space, intervals or particular statistical distributions. Within this thesis study, centroid-based clustering is used in which the clusters are defined by a central vector, which may not a member of data set. When number of clusters are fixed to \textit{k}, the approach is named as \textit{k-means clustering} and the problem is finding k cluster centers and assigning data members to nearest cluster center while minimizing the squared distances of data members to the assigned cluster centers. Although seems easy, this optimization problem is NP-hard and most of the implementations include approximate solutions. A well-known algorithm in k-means clustering is Lloyd's algorithm and it is referred as \textit{k-means algorithm} and its variation based on random initialization \textit{k-means++} are formalized in the study of Arthur and Vassilvitskii \cite{arthur2007}: 
\theoremstyle{definition}
\begin{definition}{k-means Algorithm}
\begin{enumerate}
  \item Arbitrarily choose initial $k$ centers: $C={c_1,c_2,...c_k}$
  \item For each $i \in {1,...k}$, set the cluster $C_i$ to be the set of points that are closer to $c_i$ than they are to $c_j$ for all $i\neqj$.
  \item For each $i \in {1,...k}$, set $c_i$ to be the center of mass of all points in $C_i$. $c_i=\frac{1}{|C_i|} \sum_{\substack{x\in C_i}} x$
  \item Repeat Step 2 and 3 until $C$ no longer changes.
\end{enumerate}
\end{definition}
\theoremstyle{definition}
\begin{definition}{k-means++ Algorithm}
\begin{enumerate}
  \item Take one center $c_1$, chosen uniformly from data points $X$.
  \item Take a new center $c_i$, chosen from data points $X$ with probability $\frac{D(x)^2}{\sum_{\substack{x\in X}} D(x)^2}$ where $D(x)$ is the shortest distance from a data point to the closest cluster center.
  \item Repeat Step 2 until $k$ cluster centers are selected.
  \item Proceed with the 2-4 steps of k-means Algorithm.
\end{enumerate}
\end{definition}

In this thesis study, clustering of performance analysis results is undertaken and since the number of data instances low, an approach focused on initialization is selected. Implementation of \textit{k-means++} in WEKA (Waikato Environment for Knowledge Analysis) \cite{hall2009} is used as Java API to call from ProM Framework \cite{verbeek2010prom}. 

\section{Mismatch Patterns in Process Models}
\label{sec:mismatch-patterns-in-process-models}

In cross-organizational process mining environment, there is a need to align processes of different organizations and in this scope both the organizations and their processes are similar but have significant differences. In order to align these processes and organizations, there is a need for spotting differences between process models. In the study of Dijkman \cite{dijkman2007mismatch}, a collection of patterns to describe frequent mismatches between the similar process models are presented. Mismatch patterns are grouped into three as authorization, activity and control flow mismatch patterns. Authorization mismatch patterns are based on assignment of the same tasks to different roles in different processes and left outside the scope of this thesis. Activity mismatch patterns are based on representing the tasks of a process by a different collection of activities in a different process, or not representing at all. Within the scope of this study, the related activity mismatch patterns are defined in study \cite{dijkman2007mismatch} as following:
\begin{description}
  \item[Skipped Activity] An activity exists in one process but no equivalent activity is found in the other process as illustrated in Figure~\ref{fig:skipped-activity}.
      \begin{figure}
      \centering
      \includegraphics[width=\textwidth]{3_background/mismatch-patterns-skipped-activity}
      \caption{Example of Skipped Activity Pattern}
      \label{fig:skipped-activity}
      \end{figure}
  \item[Refined Activity] An activity exists in one process but, as an equivalent, a collection of activities are existing in the other process to achieve the same task. An illustration is provided in Figure~\ref{fig:refined-activity}. 
      \begin{figure}
      \centering
      \includegraphics[width=\textwidth]{3_background/mismatch-patterns-refined-activity}
      \caption{Example of Refined Activity Pattern}
      \label{fig:refined-activity}
      \end{figure}
\end{description}
 
Control flow mismatch patterns are based on using different control-flow relations and dependencies for the same activities in different processes. Within the scope of this study, the following related control flow mismatch patterns are defined in study \cite{dijkman2007mismatch}:
\begin{description}
  \item[Activities at Different Moments in Processes] Set of activities are undertaken with different orders in different processes as shown in Figure~\ref{fig:different-moments}. 
      \begin{figure}
      \centering
      \includegraphics[width=\textwidth]{3_background/mismatch-patterns-different-moments}
      \caption{Example of Activities at Different Moments in Process Pattern}
      \label{fig:different-moments}
      \end{figure}
  \item[Different Conditions for Occurrence] Set of dependencies are same for two processes; however, occurrence condition is different. An example is provided in Figure~\ref{fig:different-conditions}.
      \begin{figure}
      \centering
      \includegraphics[width=\textwidth]{3_background/mismatch-patterns-different-conditions}
      \caption{Example of Different Conditions for Occurrence Pattern}
      \label{fig:different-conditions}
      \end{figure}
  \item[Different Dependencies] Set of activities have differ in their dependency sets. An example of this pattern is shown in Figure~\ref{fig:different-dependency}.
      \begin{figure}
      \centering
      \includegraphics[width=\textwidth]{3_background/mismatch-patterns-different-dependency}
      \caption{Example of Different Dependencies Pattern}
      \label{fig:different-dependency}
      \end{figure}
  \item[Additional Dependencies] This pattern is a special case of different dependencies where one set of activities includes the other and results with additional dependencies as illustrated in Figure~\ref{fig:additional-dependency}.
      \begin{figure}
      \centering
      \includegraphics[width=\textwidth]{3_background/mismatch-patterns-additional-dependency}
      \caption{Example of Additional Dependencies Pattern}
      \label{fig:additional-dependency}
      \end{figure}
\end{description}

As mentioned in the study \cite{dijkman2007mismatch}, their approach does not create a comprehensive list to resolve all mismatches but the most common ones spotted during case studies. In addition, from their definitions and examples it can be easily seen that these patterns are not orthogonal. Moreover, there are no algorithms provided to spot these mismatches in \cite{dijkman2007mismatch} or consequent studies, and thus implementation of spotting mismatch patterns are kept within the scope of this thesis study.
\chapter{METHODOLOGY}
\label{chp:methodology}


\section{Approach Overview}
\label{sec:approach-overview}

\section{Process Model Mining}
\label{sec:process-model-mining}
In process mining field, one of the most challenging task is to construct a process model based on the behavior in the event logs, namely process discovery. Many process discovery algorithms are proposed to address different challenges in process discovery and using different notations. However, in this thesis focus of the study is learning from the cross-organizational process mining rather than address all process discovery challenges.

 In real life, these different paths are used infrequently; however, their effect in discovery is still significant. \textit{Inductive Miner Infrequent (IMi)} is proposed in \cite{leemans2014discoveringinfrequent} as an extension to \textit{Inductive Miner} to handle noise in the event logs. By filtering the infrequent behavior, it is aimed that \textit{IMi} succeeds with improved models discovered. After each recursive step of \textit{IM}, filtering is applied if the discovered model is a flower model which represents infrequent behavior. Basic idea behind filtering is setting a user-defined threshold between 0 to 1 and with the help of this threshold, both log splittings and mining operations are done on a cleaner subset of logs. When the discovered models compared to \textit{IM}, \textit{IMi} results with lower fitness, higher precision and equal generalization.

 

\section{Performance Indicator Analysis}
\label{sec:performance-indicator-analysis}
	Replay
	Clustering

\section{Mismatch Pattern Analysis}
\label{sec:mismatch-pattern-analysis}
	BPMN Conversion


\section{Recommendation Generation}
\label{sec:recommendation-generation}



\section{Implementation}
\label{sec:implementation}

\bibliographystyle{plain}
%
% References in Bibtex format goes into below indicated file with .bib extension
\bibliography{9_references/references}
% You can use full name of authors, however most likely some of the Bibtex entries you will find, will use abbreviated first names
% If you don't want to correct each of them by hand, you can use abbreviated style for all of the references

%\appendix
%\chapter{Appendix Name}
%%%Appendix content goes here.
 
%
% If you are a Ph.D. Student you need to insert a CV at the end of you thesis
% Check vita.tex for a simple CV template in Latex
% \input{10_vita/vita.tex}
\end{document}
